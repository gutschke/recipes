\documentclass[12pt,twocolumn,a4paper]{article}
\usepackage{times}
\usepackage{helvet}
\usepackage{pifont}
\pagestyle{empty}
\textheight1.1\textheight
\parindent0pt
\parskip3pt
\title{Pumpkin \&\ Sweet Potato Soup\\ \small (K\"urbis \&\ S\"u\ss kartoffel Suppe)}
\author{Markus Gutschke}
\date{\ }
\begin{document}
\maketitle
\thispagestyle{empty}
\textit{
\begin{tabular}{ll}
$2$ Tbsp&chopped onions\\
$1$ Tbsp&olive oil\\
$\frac12$ bottle&zinfandel\\
$1$&medium size pumpkin\\
$1\frac12$&sweet potatoes\\
$1\frac12$ cups&chicken stock\\
$1$ cup&milk or cream\\
&salt\\
&ground black pepper\\
&sugar\\
&ground cinnamon\\
&ground cloves\\
&ground all-spice\\
\end{tabular}}

\leavevmode\vbox to .2in{\vss}

Chop the onions and fry them in olive oil. Keep adding a table spoon
of wine a few times, until the onions start to caramelize.

Dice the pumpkin and sweet potatoes in $\frac12$'' sized pieces. Add
them to the onions. Pour chicken broth onto vegetables and add wine
until they are fully covered.

Simmer on low heat until the vegetables are fork-tender. Blend the
vegetables. If you have, use a hand-held blender so that you do not
have to remove the vegetables from the pot; alternatively, you can
also use a food processor.

Add some milk or cream to the soup. Add spices to taste. Use only very 
little of the cinnamon, cloves, and all-spice.

\vfill\eject

\textit{
\begin{tabular}{ll}
$2$ Tl&gehackte Zwiebeln\\
$1$ Tl&Oliven\"ol\\
$\frac12$ Flasche&Ros\'e\\
$1$&mittelgrosser K\"urbis\\
$1\frac12$&S\"u\ss kartoffeln\\
$1\frac12$ Tassen&H\"uhnerbr\"uhe\\
$1$ Tasse&Milch oder Sahne\\
&Salz\\
&gemahlener schwarzer Pfeffer\\
&Zucker\\
&gemahlener Zimt\\
&gemahlene Nelken\\
&gemahlenes Allgew\"urz\\
\end{tabular}}

\leavevmode\vbox to .2in{\vss}

Die Zwiebel fein hacken und in Oliven\"ol anbraten. Mehrfach einen
L\"offel Wein zugeben bis die Zwiebeln anfangen zu karamelisieren.

Den K\"urbis und die S\"u\ss kartoffeln in zentimetergro\ss e St\"ucke 
schneiden und zu den Zwiebeln geben. Die Br\"uhe \"uber das Gem\"use
gie\ss en und mit Wein soweit auff\"ullen, da\ss\ es vollst\"andig
bedeckt ist.

Auf niedriger Flamme k\"ocheln bis das Gem\"use \hbox{weich} geworden
ist. Das Gem\"use p\"urieren. Dazu eignet sich am besten ein
Handmixstab, es kann aber auch die K\"uchenmaschine verwendet werden.

Milch oder Sahne zur Suppe geben. Mit den Gew\"urzen abschmecken. Mit
dem Zimt, den Nelken und dem Allgew\"urz sollte sparsam umgegangen
werden.

\end{document}
